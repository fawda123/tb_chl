\documentclass[letterpaper,12pt,oneside]{article}
\usepackage[paperwidth=8.5in,paperheight=11in,top=1in,bottom=1in,left=1in,right=1in]{geometry}
\usepackage{setspace}
\usepackage{times}
\usepackage{outlines}

%\linespread{2}

\begin{document}

\title{Outline of Tampa Bay regression manuscript}
\author{Marcus Beck}
\date{\today}

\maketitle

\begin{outline}
\1 Introduction
\2 Eutrophication description and its effects
\2 CWA, management, standards, need for quantitative techniques for numeric criteria
\2 Tampa Bay description, monitoring, eutrophication history, research opportunity
\2 Problems with direct interpretation of monitoring data, WTRDS approach
\2 Research goal and objectives
\3 Goal: characterize trends in chlorophyll in Tampa Bay to resolve factors that influence primary production by considering unique and interacting effects of time and freshwater inputs
\3 Obj. 1: Description of weighted regression adaptation
\3 Obj. 2: Application of model to TB dataset
\3 Obj. 3: Evaluation of model residuals with other variables 
\3 Obj. 4: Development of informed hypotheses from results
\1 Methods
\2 Study area and data
\3 Tampa Bay physical characteristics, watershed, segments
\3 Climate, ENSO, precipitation, salinity and tidal effects
\3 More detailed history of eutrophication
\3 Long-term monitoring data - overview, annual/seasonal variation, potential covariates
\3 Long-term monitoring data - used in current study, pre-processing
\2 Weighted regression
\3 Brief description of WRTDS, how it could be useful for Tampa Bay data
\3 Model functional form, mean and quantile models
\3 Model weighting and window widths
\3 Application of model to bay segments, interpolation grids, performance and comparisons 
\3 Back-transformation correction bias
\3 Salinity-normalization of prediicted values
\2 Evaluation of model residuals
\2 Co-variates used for evaluation and statistics
\2 Methods for quantifying co-variates 
\1 Results
\2 Observed trends in chlorophyll 
\3 Trends by segment - annual and decadal, peaks and likely anomaly events
\3 Trends by segment - seasonal, relationships with salinity
\2 Predicted trends in chlorophyll
\3 Model performance by type (mean, quantiles), comparison to non-weighted regression
\3 Model performance by season, annual periods
\3 Description of trends by year aggregation and normalized results, trends by decadal aggregation
\3 Evaluation of chlorophyll salinity relationships using parameters estimates over time
\2 Evaluation of model residuals
\1 Discussion
\2 Primary conclusions
\3 Weighted regression approach is useful for estuaries
\3 Descriptions of water quality trends should consider heterogeneous variance
\3 Use of quantile models to describe response not conditional on mean is critical
\2 Description of change provided by weighted regression
\3 Relationships of chlorophyll and salinity is dynamic over time, similar to Hirsch results for streams
\3 Model provided evidence of shifts in pollutant sources over time, similar to Hirsch
\3 Utility of normalized results and importance of identifying non-stationarity
\3 Importance of quantile models, relation to criteria development
\2 Limitation and future applicaitons
\3 Evaluation of model with other spatial/temporal scales
\3 Need for uncertainty estimates and evaluation of window widths
\3 Lack of relationships between covariates and model residuals
\1 Conclusions
\2 Re-iteration of conclusions
\2 Informed hypotheses from results
\3 Temporal dynamics and changes in relationships of chlorophyll with salinity point to pollutant sources and physical forcing factors (flushing)
\end{outline}

\end{document}